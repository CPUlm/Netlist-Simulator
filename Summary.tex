\documentclass[french, 12pt]{article}
\usepackage[a4paper, top=1cm, bottom=1cm, left=1cm, right=1cm]{geometry}

\usepackage{lmodern}
\usepackage[T1]{fontenc}
\usepackage[french]{babel}
\usepackage{grammar}

\usepackage{parskip}
\usepackage{hyperref}
\usepackage{caption}
\usepackage{subcaption}
\usepackage{graphicx}

\pagenumbering{gobble}

\title{Projet NetList}
\author{Gabriel Desfrene}
\date{}

\begin{document}
\maketitle

Voici mes quelques notes pour la phase 1 du projet ``Microprocessor'' du cours Systèmes numériques.
Le code associé à ce projet est disponible sur un dépôt GitHub à l'adresse suivante :
\begin{center}
      \url{https://github.com/desfreng/netlist/tree/gabriel}
\end{center}

\section {Langage NetList}\label{sec:langage-netlist}
Le langage que le simulateur accepte possède quelques extensions par rapport à celui proposé à l'origine.
Ces extensions comprennent :

\begin{itemize}
      \item Toutes les opérations sont étendues à des bus.
      \item Les constantes peuvent être données en base 10 à l'aide de la syntaxe
            suivante :
            \[\verb|0d|\langle\text{valeur en base 10}\rangle\verb|:|\langle\text{taille}\rangle\]
      \item De la même manière, les constantes peuvent être données en base 16 en
            préfixant leur valeur par \verb|0x|.
      \item Par commodité, les constantes données en base 2 peuvent être préfixées
            par \verb|0b|.
      \item Il est possible d'ajouter des commentaires dans les fichiers d'entrée.
            Ceux-ci commencent par \verb|#| et s'étalent sur toute la fin de la ligne.
\end{itemize}


Une grammaire du langage accepté est donnée en \autoref{fig:grammar}.
Des restrictions ont également été posées afin de garantir une consistance sur la taille des bus manipulés.
Ainsi :


\begin{figure}[htp]
      \begin{grammar}
            \symb{program} \is \kw{INPUT} \symb{ref-list} \kw{OUTPUT} \symb{ref-list} \kw{VAR} \symb{decl-list} \kw{IN} \symb{eqs} \eoi \si

            \symb{ref-list} \is \symb{ref-list-non-empty} \linealt \eps \si

            \symb{decl-list} \is \symb{decl-list-non-empty} \linealt \eps \si

            \symb{ref-list-non-empty} \is \symb{ident} \kw{,} \symb{ref-list-non-empty} \linealt \symb{ident} \si

            \symb{decl-list-non-empty} \is \symb{ident} \symb{size-spec} \kw{,} \symb{decl-list-non-empty} \linealt \symb{ident} \symb{size-spec} \si

            \symb{eqs} \is \symb{ident} \kw{=} \symb{expr} \symb{eqs} \linealt \eps \si

            \symb{expr} \is \symb{arg}
            \alt            \kw{NOT}    \symb{arg}
            \alt            \kw{AND}    \symb{arg}      \symb{arg}
            \alt            \kw{NAND}   \symb{arg}      \symb{arg}
            \alt            \kw{OR}     \symb{arg}      \symb{arg}
            \alt            \kw{XOR}    \symb{arg}      \symb{arg}
            \alt            \kw{MUX}    \symb{arg}      \symb{arg}      \symb{arg}
            \alt            \kw{REG}    \symb{ident}
            \alt            \kw{CONCAT} \symb{arg}      \symb{arg}
            \alt            \kw{SELECT} \symb{bus-size} \symb{arg}
            \alt            \kw{SLICE}  \symb{bus-size} \symb{bus-size} \symb{arg}
            \alt            \kw{ROM}    \symb{bus-size} \symb{bus-size} \symb{arg}
            \alt            \kw{RAM}    \symb{bus-size} \symb{bus-size} \symb{arg}  \symb{arg}  \symb{arg}  \symb{arg}
            \si

            \symb{arg}  \is \symb{ident} \linealt \symb{bin-digits} \linealt \symb{bin-const} \linealt \symb{dec-const} \linealt \symb{hex-const} \si

            \symb{ident}        \is [\kw{a}-\kw{z}\kw{A}-\kw{Z}\kw{\_}][\kw{a}-\kw{z}\kw{A}-\kw{Z}\kw{0}-\kw{9}\kw{\_}]$\ast$ \si

            \symb{bin-digits}    \is [\kw{0}\kw{1}][\kw{0}\kw{1}]$\ast$ \si

            \symb{bin-const}    \is \kw{0b}[\kw{0}\kw{1}][\kw{0}\kw{1}]$\ast$ \symb{size-spec} \si

            \symb{dec-const}    \is \kw{0d}[\kw{0}-\kw{9}][\kw{0}-\kw{9}]$\ast$ \kw{:} \symb{bus-size} \si

            \symb{hex-const}    \is \kw{0x}[\kw{0}-\kw{9}\kw{a}-\kw{f}][\kw{0}-\kw{9}\kw{a}-\kw{f}]$\ast$ \symb{size-spec} \si

            \symb{size-spec}    \is \kw{:} \symb{bus-size} \linealt \eps \si

            \symb{bus-size}     \is [\kw{1}-\kw{9}][\kw{0}-\kw{9}]$\ast$ \si
      \end{grammar}
      \caption{Grammaire acceptée par le simulateur pour les fichiers NetList. Axiome : \textit{program}.}
      \label{fig:grammar}
\end{figure}

\begin{itemize}
      \item Les opérations logiques binaires \textbf{AND}, \textbf{NAND},
            \textbf{OR} et \textbf{XOR} doivent avoir des arguments de même taille.
      \item Les deux derniers arguments de l'opération \textbf{MUX} doivent être
            de même taille.
      \item Les opérations \textbf{SLICE} et \textbf{SELECT} doivent avoir des
            indices de bus cohérents avec la taille de l'argument.
\end{itemize}


Dans le cas où ces contraintes ne seraient pas respectées, le parseur codé refusera l'entrée.
De plus, le parseur vérifiera que chaque équation est ``bien formée'', c'est-à-dire que les deux membres ont bien la même taille.


La taille des arguments est déterminée de la manière suivante :

\begin{itemize}
      \item La taille d'une variable est celle donnée lors de sa
            déclaration dans la section \textbf{VAR}.
      \item La taille d'une suite de chiffres 0 et 1 est la longueur de cette
            suite.
      \item Pour une constante donnée en base 10, la taille est nécessairement
            donnée après sa valeur suivie de ``\verb|:|''.
      \item Pour les constantes binaires et hexadécimales, la taille peut être
            indiquée de la même manière que pour la base 10. Si aucune indication
            n'est fournie, la taille est inférée à partir de la longueur de la
            suite de chiffres.
\end{itemize}


Si la taille donnée ne permet pas de représenter la valeur associée, le
parseur générera une erreur.


\section {Fonctionnalités}


Le simulateur est doté de plusieurs fonctionnalités permettant la manipulation des NetList.
Celles-ci sont sélectionnées lors de l'exécution via les arguments donnés au programme.
Une description des combinaisons acceptées est disponible dans l'aide du programme.
Pour l'obtenir, ajoutez l'argument \verb|--help| en premier.
Voici les fonctionnalités disponibles :


\begin{itemize}
      \item L'affichage de la NetList parsée sur la sortie standard.
      \item L'export de la NetList sous la forme d'un graphe.
      \item L'affichage du résultat de l'ordonnanceur.
      \item La simulation de la NetList.
\end{itemize}

\subsection{Export en graphe}

Lors de l'exportation de la NetList sous forme de graphe, le fichier NetList donné en argument est converti en un graphe
au format \href{https://graphviz.org/}{\textit{Graphviz}} sur la sortie standard.
Vous pouvez utiliser un outil tel que \verb|dot| pour exporter le graphe au format PDF.
La \autoref{fig:graphs} donne des exemples de graphes que retourne le programme.
Les conventions suivantes sont utilisées lors de l'export en graphe :
\begin{itemize}
      \item Un n\oe{}ud ayant un nom en \textbf{gras} représente une variable d'entrée.
      \item Un n\oe{}ud de forme rectangulaire représente une sortie.
      \item Les arcs pleins représentent les dépendances entre les variables de la
            NetList. Ces dépendances sont prises en compte par l'ordonnanceur.
      \item Les arcs en pointillés représentent les dépendances non prises en
            compte par l'ordonnanceur, telles que les dépendances induites par une
            opération \textbf{REG} ou \textbf{RAM} lors de l'écriture de la valeur
            en mémoire.
\end{itemize}

\begin{figure}[htp]
      \centering
      \begin{subfigure}{.62\textwidth}
            \begin{center}
                  \includegraphics[height=200px]{./out/ful.dot.pdf}
            \end{center}
            \caption{\texttt{fulladder.net}}
            \label{fig:sub1}
      \end{subfigure}%
      \begin{subfigure}{.38\textwidth}
            \begin{center}
                  \includegraphics[height=200px]{./out/cm2.dot.pdf}
            \end{center}
            \caption{\texttt{cm2.net}}
            \label{fig:sub2}
      \end{subfigure}
      \caption{Quelques graphes produits par le programme.}
      \label{fig:graphs}
\end{figure}

\subsection{La simulation}

Pour simuler les NetList, certaines contraintes ont été définies :
\begin{itemize}
      \item Les blocs \textbf{ROM}, identifiés par la variable dont ils
            déterminent la valeur, doivent nécessairement être initialisés avant
            la simulation. Le programme reportera une erreur dans le cas contraire.
      \item Les blocs \textbf{RAM}, identifiés par la variable dont ils
            déterminent la valeur, peuvent être initialisés avant la simulation.
            Dans le cas où aucune valeur n'est donnée, le programme affichera
            un avertissement et la mémoire est initialisé à $0$.
      \item La première valeur des registres est fixée à $0$.
\end{itemize}

La \autoref{fig:input-grammar} donne le format requis pour les fichiers de l'entrée
qui initialisent les blocs de mémoire.

\begin{figure}[htp]
      \begin{grammar}
            \symb{input-file} \is \symb{block-list} \eoi \si

            \symb{block-list} \is \symb{block-list-non-empty} \linealt \epsilon \si

            \symb{block-list-non-empty} \is \symb{block-def} \symb{block-list-non-empty} \linealt \symb{block-def} \si

            \symb{block-def} \is \symb{ident} \kw{:} \symb{bus-size} \kw{=} \kw{[} \symb{value-list} \kw{]} \si

            \symb{value-list} \is \symb{value} \kw{,} \symb{value-list} \linealt \symb{value} \si

            \symb{value} \is \symb{bin-digits} \linealt \symb{bin-const} \linealt \symb{dec-const} \linealt \symb{hex-const} \si
      \end{grammar}
      \caption{Grammaire acceptée par le simulateur pour les fichiers d'entrée. Axiome : \textit{input-file}.}
      \label{fig:input-grammar}
\end{figure}

Ainsi, pour le fichier \verb|ram.net|, la structure suivante convient afin
d'initialiser la mémoire de l'opération \textbf{RAM} :
\begin{verbatim}
o : 4 = [ 0xf, 0b1001, 0d3:4, 0110 ]
\end{verbatim}


Si aucune limite de cycles n'est spécifiée, le programme s'arrête lorsqu'il
reçoit un \textit{SIGTERM} ou un \textit{SIGINT}.


\section{Difficultés}

Lors de la réalisation de ce projet, quelques difficultés ont été rencontrées :

\begin{itemize}
      \item L'apprentissage des versions de C++ récentes ne s'est pas
            fait en douceur. Ce projet a permis d'apprendre et de manipuler des
            constructions en C++ plus récentes telles que les \textit{templates}%
            \footnote{Je n'avais pas fait de C++ depuis \textbf{très} longtemps\dots}.
      \item La mise au propre de la grammaire du langage NetList parsée a
            également posé quelques soucis. Puisque le parseur a été
            réimplémenté en faisant attention à la cohérence du fichier donné en
            entrée, des problèmes sont apparus pour certaines opérations telles que
            \textbf{CONCAT} ou \textbf{SLICE} lorsqu'elles prennent des
            constantes comme argument.
      \item Il a été difficile\footnote{Voir la liste des commits.} de faire en
            sorte que MSVC accepte de compiler ce projet.
            La mise en place de builds automatiques a beaucoup aidé à
            vérifier que ce projet compile dans de bonnes conditions.
\end{itemize}

\end{document}